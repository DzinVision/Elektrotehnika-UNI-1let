\documentclass[a4paper, 12pt]{article}
\usepackage[slovene]{babel}
\usepackage[utf8]{inputenc}
\usepackage[T1]{fontenc}
\usepackage{hyperref}
\usepackage{graphicx}
\usepackage{amssymb}
\usepackage{amsmath}
\usepackage{mathtools}
\setlength{\parindent}{0px}
\setlength{\parskip}{10px}
\usepackage{enumitem}
\usepackage{blindtext}
\usepackage{lipsum}
\usepackage{scrextend}
\usepackage{tikz}
\usepackage[margin=1in]{geometry}
\usetikzlibrary{shapes,arrows}
\usepackage{listings}
\usepackage{color}

\title{Zapiski iz pouka Osnove programiranja II Programiranje Mikrokontrolerjev}
\author{Matej Blagšič}

\begin{document}

%---------------------------------------------------za diagrame
\tikzstyle{decision} = [diamond, draw, fill=blue!20, 
text width=4.5em, text badly centered, node distance=3cm, inner sep=0pt]
\tikzstyle{block} = [rectangle, draw, fill=blue!20, 
text width=5em, text centered, rounded corners, minimum height=4em]
\tikzstyle{line} = [draw, -latex']
\tikzstyle{cloud} = [draw, ellipse,fill=red!20, node distance=3cm,
minimum height=2em]
\tikzstyle{nothing} = [draw, node distance =3cm]

%-------------------------------------za programiranje text barvanje
\definecolor{dkgreen}{rgb}{0,0.6,0}
\definecolor{gray}{rgb}{0.5,0.5,0.5}
\definecolor{mauve}{rgb}{0.58,0,0.82}
\definecolor{lgray}{RGB}{250,250,250}
\lstset{
	frame=l,%single
	language=C,
	aboveskip=3mm,
	belowskip=3mm,
	showstringspaces=false,
	columns=flexible,
	basicstyle={\small\ttfamily},
	numbers=none,
	numberstyle=\tiny\color{gray},
	keywordstyle=\color{blue},
	commentstyle=\color{dkgreen},
	stringstyle=\color{mauve},
	breaklines=true,
	breakatwhitespace=true,
	tabsize=4,
	backgroundcolor=\color{lgray},
	moredelim=**[is][\color{dkgreen}]{@}{@}
}
%---------------------------------------//
\maketitle
\thispagestyle{empty}
\pagebreak
\setcounter{page}{1}

\tableofcontents
\pagebreak



\section{Osnovno}

Pri temu predmetu bomo obravnavali jezik C. Za uporabo lahko preneseš okolje Codeblocks z MinGW inštalacijo ali posebej MinGW compiler in poljubno okolje(Jetbrains).\

Pomembno je, da imaš predznanje iz prejšnjega polletja pri Javascriptu, saj so tipi spremenljivk, sintaksa in drugo zelo podobno, tako da v detajle o stvarih, ki so enake ne bom šel.\

Vsak dokument začnemo z \framebox{\lstinline|@#include <stdio.h>@|\par} za standardne vhodne in izhodne ukaze.
Vsaka koda se izvaja znotraj main funkcije:


\begin{lstlisting}
int main(){
    printf("Hello!\n");
    return 0;
}
\end{lstlisting}

\textbf{Prav tako je pomembno uporabiti PODPIČJE za vsakim ukazom/vrstico!!!}

Če začnemo na začetku, opazimo \texttt{\#include} ukaz. Ta se izvrši, preden se karkoli drugega. V temu primeru lahko vnesemo knjižnice. Te nam olajšajo programiranje s tem, da nam en ukaz izvede več ukazov, ki bi jih morali tipkati na roke. To datoteko/knjižnico navedemo lahko z "datoteka" navednicam. Če pa damo v <datoteka>, potem pa išče datoteke v sistemskih mapah okolja. Te datoteke so vrste \textbf{header} s končnico \textbf{.h}. V našem primeru je knjižnica za pisat in brat podatke - vhodne in izhodne podatke.\\
To je podobno kot v javascriptu:  \lstinline|<script src="datoteka">|\ 
\section{Podatkovni tipi}
Si poglejmo zgled:
\begin{lstlisting}
int main(){
	int a;
	float b; //spremenljivka

	printf("Vprisi prvo vrednost");
	scanf("%d", &a);
	printf("Vpisi drugo vrednos");
	scanf("%f", &b);
	printf("%d + %f = %f\n", a, b, a+b);
	return 0;
}
\end{lstlisting}

C je občutljiv na tip podatkov. To pomeni, da moramo vrsto podatka navesti.
To pomeni, da se moramo sami odločiti, kakšen tip podatka bo nosila spremenljivka.\\
Vemo, da v Javascriptu nismo rabili napisati tipa spremenljivke, le \texttt{var}.

\pagebreak

\subsection*{Tipi spremenljivk:}

\begin{description}[align=left, labelwidth=2cm]
	\item [int] Intager: Celoštevilski tip. Uporabimo formatno določilo \%d
	\item [float] Float: Realna števila. Uporabimo formatno določilo \%f
	\item [double] Double: Double precision Float: "dvojni float"
	\item[char] Char: 1 byte velik karakter
\end{description}



\section{Branje podatkov}

Da nam program prebere podatek, uporabimo funkcijo:

{\centering\framebox{\lstinline|scanf("formatno\_dolocilo", \&spremenljivka);|}\par}

Vidimo, da moramo najprej deklarirati tip podatka, ki ga pričakuje operator Scanf. Potem pa določimo naslovni operator \& in nato za njim spremenljivko, ki naj sprejme podatek.

\section{Pisanje podatkov}

Za pisanje podatkov uporabimo funkcijo:

{\centering\framebox{\lstinline|printf("formatni\_niz", izrazi)|}\par}

Pomembne so tudi ubežne sekvence. To so \texttt{\textbackslash r \textbackslash n \textbackslash t}, ki povejo, kaj se zgodi, ko se text izpiše. \texttt{\textbackslash n} naredi novo vrstico(new line) po besedilu, \texttt{\textbackslash t} je tabulator...\

Tako v našem primeru, se a izpiše tam, kjer je njegov \%d in b, kjer je \%f ter vsota a+b tam, kjer je \%f(glej izsek programske kode).

\section{Funkcije}

Funkcije deklariramo:\

{\centering\framebox{\underline{\lstinline|float|} \lstinline| imeFunkcije()\{/*telo funkcije*/	return 0;\}|}\par}

Opazimo, da funkcijo deklariramo kot float oz. funkcijo, ki vrne realno število. V resnici lahko funkcije definiramo kot karkoli hočemo, glede na to, kaj naj bi vrnila.\

Prav tako vidimo, da glavna zanka, v kateri se koda izvaja, je \texttt{main}. v tej kodi se izvajajo vsi programi in funkcije. Tako se koda, ki je napisana tu notri, se prevede in spremeni v izvršilno kodo(executable).


\end{document}